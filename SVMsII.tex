\section{SVMs}
\setcounter{problem}{0}

Consider a linear two-class SVM classifier defined by the parameters 
$w=[3;2]$ and $b=2$. Answer the following questions providing adequate 
motivations:\bigskip

\Problem{Is the point $x_1=[3;-1]$ classified according to the trained SVM 
as positive?}\medskip

\Solution{Classifying the point according to sign 
$f(\underline{x})=\mathrm{sign}(\underline{w}^T \underline{x} + b)$, it will be 
classified as 
positive only if $\underline{w}^T \underline{x} + b\geq 1$. Since 
$\underline{w}^T \underline{x_1}+ b = 3\cdot 3 -2 \cdot 1 + 
2 = 9 \geq 1$, $\underline{x_1}$ is classified as positive.}\vspace{.4cm}

\Problem{Assume to collect a new sample $x_2=[-2;2]$ with positive label. 
Do you need to retrain the SVM?}\medskip

\Solution{The sample lie on the hyperplane since
$\underline{w}^T \underline{x_2}+ b = 3\cdot (-2)+ 2 \cdot 2 + 2 = 0$.
In this case it is necessary to retrain the SVM to find another hyperplanes 
that correctly classified this data-point.}\vspace{.4cm}

\Problem{Does the answer to the previous question apply to a generic new 
sample?}\medskip

\Solution{No. If a point is correctly classified it is not necessary since this 
sample do not influence the model. However, if it is misclassified or a 
support-vector (whether it is on the hyperplane or within the margin), it is 
necessary to retrain the model taking into consideration the new training 
sample collected.}\vspace{.4cm}
