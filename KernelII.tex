\section{Kernels II}

Explain in each case if you would apply the kernel trick to represent your 
data. Which kind of kernel would you use? Why? If relying on kernels is not 
a good choice, which other transformation would you use?
\begin{itemize}
	\item[(a)] In this case I would apply the kernel trick because the data are 
	linearly non-separable. %FIXME
	A Gaussian kernel or a Radial Basis Function could be suitable choices for 
	the data(?).
	\item[(b)] In this case I would apply the kernel trick because the data are 
	not linearly separable.  
	radius of a circle centred in $(0, 0)$
	\item[(c)] In this case, the points are linearly separable, so it is not 
	necessary to apply the kernel trick.  
	\item[(d)] In this case I would apply the kernel trick because the data 
	are linearly non-separable. Moreover, relying on kernels is not 
	a good choice in this case, \\
	\href{https://www.researchgate.net/publication/309545761\_A\_Structure- 
	Adaptive\_Hybrid\_RBF-BP\_Classifier\_with\_an\_Optimized\_Learning\_Strategy}{Adaptive
	 Hybrid RBF-BP Classifier with an Optimized Learning Strategy}
\end{itemize}
