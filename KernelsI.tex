\section{Kernels I}

Consider $\mathbf{x}$, $\mathbf{y} \in \mathbb{R}^d$, $d \in \mathbb{N}$. 
Explain in detail why each of the following functions is or is not a valid 
kernel:\bigskip

\Problem{$K(x,y)=x^Ty+(x^Ty)^2$}\medskip

\Solution{
First of all, it is known that the linear kernel $x^Ty$ is a valid kernel. 
Taking advantage of these two rules:
\begin{enumerate}
	\item $K(x,y)=q (K_1(x,y))$ where $q$ is a polynomial with 
	non-negative coefficient,
	\item $K(x,y)=K_1(x,y)+K_2(x,y)$
\end{enumerate} 	
it is proved that using the polynomial rule, the second term $(x^Ty)^2$ is a 
valid kernel. Applying the sum rule, is reached the conclusion that the 
following is a valid kernel:
\begin{equation*}
K(x,y) = x^Ty+(x^Ty)^2  \mbox{.}
\end{equation*}
}\vspace{-0.3cm}

\Problem{$K(x,y)=x^2e^{-y}$, $d=1$}\medskip

\Solution{
In this case, it can be proved that the expression $x^2e^{-y}$ is not 
symmetric.  
In fact, since $d=1$, $x$ and $y$ are scalars. So, choosing two integers, for 
example $x=1$ and $y=0$, we obtain that $x^2e^{-y}=1e^0=1 \neq y^2e^{-x}=0$, so 
$K(x,y)=x^2e^{-y}$ cannot be a valid kernel.
}\vspace{.5cm}

\Problem{$K(x,y)=ck_1(x,y)+k_2(x,y)$, where $k_1(x,y)$, $k_2(x,y)$ are 
	valid kernels in $\mathbb{R}^d$}\medskip

\Solution{
It is already known that $k_1(x,y)$ and $k_2(x,y)$ are valid kernels.
Since multiplying a valid kernel by a constant it remains a valid kernel, and 
since the sum of two valid kernels produces a valid kernel, taking advantage of 
these two rules: 
\begin{enumerate}
	\item $K(x,y)=c \times K_1(x,y) \quad \mathrm{ with } \: c > 0 \: \mathrm{ 
	constant }$,
	\item $K(x,y)=K_1(x,y)+K_2(x,y)$
\end{enumerate} 
it is proved that, only if the constant $c$ is positive, the kernel
\begin{equation*}
K(x,y)=ck_1(x,y)+k_2(x,y)
\end{equation*}
is a valid kernel.
}
